%%%%%%%%%%%%%%%%%%%%%%%%%%%%%%%%%%%%%%%%%%%%%%%%%%%%%%%%%%%%
%%% LIVECOMS ARTICLE TEMPLATE FOR BEST PRACTICES GUIDE
%%% ADAPTED FROM ELIFE ARTICLE TEMPLATE (8/10/2017)
%%%%%%%%%%%%%%%%%%%%%%%%%%%%%%%%%%%%%%%%%%%%%%%%%%%%%%%%%%%%
%%% PREAMBLE
\documentclass[9pt,review]{livecoms}
% Use the 'onehalfspacing' option for 1.5 line spacing
% Use the 'doublespacing' option for 2.0 line spacing
% Use the 'lineno' option for adding line numbers.
% Use the "ASAPversion' option following article acceptance to add the DOI and relevant dates to the document footer.
% Use the 'pubversion' option for adding the citation and publication information to the document footer, when the LiveCoMS issue is finalized.
% The 'bestpractices' option for indicates that this is a best practices guide.
% Omit the bestpractices option to remove the marking as a LiveCoMS paper.
% Please note that these options may affect formatting.

\usepackage{lipsum} % Required to insert dummy text
\usepackage[version=4]{mhchem}
\usepackage{siunitx}
\DeclareSIUnit\Molar{M}
\usepackage[italic]{mathastext}
\graphicspath{{figures/}}

%%%%%%%%%%%%%%%%%%%%%%%%%%%%%%%%%%%%%%%%%%%%%%%%%%%%%%%%%%%%
%%% IMPORTANT USER CONFIGURATION
%%%%%%%%%%%%%%%%%%%%%%%%%%%%%%%%%%%%%%%%%%%%%%%%%%%%%%%%%%%%

\newcommand{\versionnumber}{0.1}  % you should update the minor version number in preprints and major version number of submissions.
\newcommand{\githubrepository}{\url{https://github.com/openforcefield/protein-benchmark-data}}  %this should be the main github repository for this article

%%%%%%%%%%%%%%%%%%%%%%%%%%%%%%%%%%%%%%%%%%%%%%%%%%%%%%%%%%%%
%%% ARTICLE SETUP
%%%%%%%%%%%%%%%%%%%%%%%%%%%%%%%%%%%%%%%%%%%%%%%%%%%%%%%%%%%%
\title{Experimental datasets for benchmarking protein force fields [Article v\versionnumber]}

\author[1*]{Firstname Middlename Surname}
\author[1,2\authfn{1}\authfn{3}]{Firstname Middlename Familyname}
\author[2\authfn{1}\authfn{4}]{Firstname Initials Surname}
\author[2*]{Firstname Surname}
\affil[1]{Institution 1}
\affil[2]{Institution 2}

\corr{email1@example.com}{FMS}  % Correspondence emails.  FMS and FS are the appropriate authors initials.
\corr{email2@example.com}{FS}

\orcid{Author 1 name}{AAAA-BBBB-CCCC-DDDD}
\orcid{Author 2 name}{EEEE-FFFF-GGGG-HHHH}

\contrib[\authfn{1}]{These authors contributed equally to this work}
\contrib[\authfn{2}]{These authors also contributed equally to this work}

\presentadd[\authfn{3}]{Department, Institute, Country}
\presentadd[\authfn{4}]{Department, Institute, Country}

\blurb{This LiveCoMS document is maintained online on GitHub at \githubrepository; to provide feedback, suggestions, or help improve it, please visit the GitHub repository and participate via the issue tracker.}

%%%%%%%%%%%%%%%%%%%%%%%%%%%%%%%%%%%%%%%%%%%%%%%%%%%%%%%%%%%%
%%% PUBLICATION INFORMATION
%%% Fill out these parameters when available
%%% These are used when the "pubversion" option is invoked
%%%%%%%%%%%%%%%%%%%%%%%%%%%%%%%%%%%%%%%%%%%%%%%%%%%%%%%%%%%%
\pubDOI{10.XXXX/YYYYYYY}
\pubvolume{<volume>}
\pubissue{<issue>}
\pubyear{<year>}
\articlenum{<number>}
\datereceived{Day Month Year}
\dateaccepted{Day Month Year}

%%%%%%%%%%%%%%%%%%%%%%%%%%%%%%%%%%%%%%%%%%%%%%%%%%%%%%%%%%%%
%%% ARTICLE START
%%%%%%%%%%%%%%%%%%%%%%%%%%%%%%%%%%%%%%%%%%%%%%%%%%%%%%%%%%%%

\begin{document}

\begin{frontmatter}
\maketitle

\begin{abstract}
250 word limit
\end{abstract}

\end{frontmatter}

\section{Introduction}

\begin{itemize}

\item Background

	\begin{itemize}
	\item Role of molecular dynamics in understanding protein structure and function and in drug design
	\item Brief history of protein force fields
	\end{itemize}

\item Gap in field

	\begin{itemize}
	\item Force fields are trained against different empirical targets and are expected to describe some behaviors well and others poorly
	\item Force fields for proteins often borrow parameters from more general force fields that aim to describe broader chemistry
	\item Need for a comprehensive collection of experimental datasets that interrogates a wide range of physical properties of proteins
	\end{itemize}

\item Goals of current review

	\begin{itemize}
	\item Description of available datasets and not prescription of how comparisons should be made
	\item Focus on peptides and globular proteins without ligands or cofactors to narrow scope
	\end{itemize}

\item Target audience

	\begin{itemize}
	\item Researchers involved in developing or assessing protein force fields
	\item Assume familiarity with molecular dynamics techniques, force field terms, and basics of protein structure
	\end{itemize}

\item Review format

	\begin{itemize}
	\item Explanation of Perpetual Review format
	\item Instructions for community involvement
	\end{itemize}

\item Outline of review sections

\end{itemize}

\section{Goals of benchmark datasets}

\begin{itemize}

\item Target observables

	\begin{itemize}
	\item Target experimental observables instead of structural models or quantum chemistry data
	\end{itemize}

\item Accessibility

	\begin{itemize}
	\item Identify datasets that are accessible without paywalls or restrictive licenses
	\end{itemize}

\item Multiple scales

	\begin{itemize}
	\item Identify observables that interrogate physical properties at different length and time scales
	\item Goal is to assess force fields rather than train parameters, so computational cost can be high
	\item System size should range from small---small enough to sample an ensemble exhaustively---to medium---large enough to exhibit stable folding behaviors
	\end{itemize}

\item Discriminatory power

	\begin{itemize}
	\item Identify systems that can discriminate between force fields
	\item For example, most protein force fields can describe lysozyme well
	\end{itemize}

\end{itemize}

\section{Room-temperature (RT) crystallography}

\begin{itemize}

\item Advantages of RT crystals

	\begin{itemize}
	\item RT crystals are higher quality and exhibit lower mosaicity than low-temperature crystals
	\item Proteins in RT crystals fluctuate more than those in low temperature crystals
	\item Observables are accessible in public databases in a common format
	\end{itemize}

\item Observables

	\begin{itemize}

	\item Electron density

		\begin{itemize}
		\item Electron density is independent of a structural model unless molecular replacement was used to solve phases
		\item Electron density from solvent molecules can be included
		\item Comparing simulations to experiments

			\begin{itemize}
			\item Quality metrics for structural models, e.g. R-factors or correlation coefficients, are likely too sensitive to meaningfully discriminate between force fields
			\item Differences can be visualized by an $F_O - F_C$ map
			\item A quantitative metric is a comparison between a structural model refined against simulated electron density and a structural model refined against experimental density, e.g. an RMSD
			\end{itemize}

		\end{itemize}

	\item Reflections

		\begin{itemize}
		\item Raw reflections are totally independent of a structural model
		\item Reflections are available in PDB entries
		\item Non-Bragg peaks from diffuse scattering inform on large-scale fluctuations
		\end{itemize}

	\item Debye-Waller (B) factors

		\begin{itemize}
		\item B factors are available in PDB entries
		\item B factors inform on local flexibility
		\item A drawback is that B factors may reflect disorder in the crystal lattice rather than flexibility of the crystallized molecules
		\end{itemize}
	
	\item Populations of alternative conformations

		\begin{itemize}
		\item Although alternative conformations rely on a structural model, this low resolution metric may discriminate between force fields that perform similarly on other observables
		\end{itemize}

	\end{itemize}

\item Running crystal simulations

	\begin{itemize}
	\item Simulations of single unit cells are less expensive but may miss fluctuations that are important for some observables
	\item Simulation of supercells are more realistic but may fail to maintain the correct symmetry
	\item May need to include co-solvents in mother liquor
	\end{itemize}

\item Systems

	\begin{itemize}

	\item Criteria/desiderata

		\begin{itemize}
		\item High resolution ($<= \SI{1.2}{\angstrom}$) crystals to ensure high quality target data and identify tautomers and protonation states
		\item Protonation state can be determined unambiguously by neutron diffraction
		\item Aim for diversity in secondary structure
		\item Systems for which data from multiple crystals with different symmetry are available are useful
		\end{itemize}

	\item Systems

		\begin{itemize}
		\item David Case
		\item Julian Chen
		\item James Fraser
		\item Daniel Keedy
		\item Michael Wall
		\end{itemize}
	
	\end{itemize}

\end{itemize}

\begin{table*}[t]
\caption{Room-temperature crystallography datasets}
\begin{tabular}{p{0.3 \linewidth} p{0.1 \linewidth} p{0.2 \linewidth} p{0.15 \linewidth} p{0.15 \linewidth}}
\toprule
Description & PDB ID & Experiments & Experimental references & Computational references \\
\midrule
Endoglucanase & 3X2P & X-ray diffraction &  &  \\
& & Neutron diffraction & & \\
Scorpion toxin II & 1AHO & X-ray diffraction &  &  \\
\bottomrule
\end{tabular}
\label{tab:xtal}
\end{table*}

\section{Nuclear magnetic resonance spectroscopy}

\begin{itemize}

\item Advantages of NMR

	\begin{itemize}
	\item NMR experiments are performed in the desired ensemble for most applications
	\item Comparison to NMR data may reveal native state bias that is difficult to diagnose with crystal simulations
	\item Many NMR observables can be related to specific FF terms
	\end{itemize}

\item Observables

	\begin{itemize}

	\item Chemical shift

		\begin{itemize}
		\item Easily accessible for many systems in BMRB
		\item Directly informs on local backbone conformation for unstructured peptides and disordered proteins
		\item Difficult to interpret for larger, folded proteins due to aromatic ring currents, spin diffusion, etc.
		\end{itemize}

	\item Scalar coupling

		\begin{itemize}
		\item Scalar coupling values for backbone amide proton inform on local backbone conformation
		\item Requires Karplus parameters, which can be derived from QM
		\end{itemize}

	\item Helical propensities (merge with chemical shift section?)

		\begin{itemize}
		\item \ce{^{13}C=O} chemical shifts inform on helical propensities of amino acids
		\item Benchmarks can target chemical shifts directly or Lifson-Roig helix extension parameters
		\end{itemize}

	\item Nuclear Overhauser effect (NOE) spectroscopy

		\begin{itemize}
		\item NOEs inform on interactions between residues distant in primary sequence
		\item NOE intensities are nonlinear averages that are difficult to converge, so they may serve better as ordinal (i.e. strong/medium/weak) rather than quantitative assessments
		\end{itemize}

	\item Residual dipolar coupling (RDC)

		\begin{itemize}
		\item RDCs inform on large spatial motions
		\item Calculating RDCs for large proteins requires computing an expensive alignment tensor
		\end{itemize}

	\item Spin relaxation

		\begin{itemize}
		\item Spin relaxation rates inform on large spatial motions for folded proteins
		\item Spin relaxation can discriminate between force fields that describe global conformations and those that describe only local conformations
		\item There is error from zero point motion and difference between modeled and true bond lengths, but the necessary correction may be small enough to ignore
		\item Spin relaxation rates will be difficult to converge for large, folded proteins
		\end{itemize}

	\end{itemize}

\item Running NMR simulations

	\begin{itemize}
	\item Viscosity of water model is known to affect tumbling rates and thus spin relaxation rates
	\end{itemize}

\item Systems

	\begin{itemize}
	\item Kyle Beauchamp chemical shifts and scalar couplings
	\item Bernie Brooks spin relaxation dataset for lipids, good for methods
	\item Lillian Chong scalar couplings for protein mimetics, good for methods
	\item Kresten Lindorff-Larsen chemical shift and NOEs
	\item Samuli Ollila spin relaxation dataset for proteins
	\item Paul Robustelli chemical shifts, NOEs, and helical propensities
	\item Lars Sch\"{a}fer c-Myb chemical shifts and NOEs
	\end{itemize}

\end{itemize}

\begin{table*}[t]
\caption{Nuclear magnetic resonance spectroscopy datasets}
\begin{tabular}{p{0.3 \linewidth} p{0.1 \linewidth} p{0.2 \linewidth} p{0.15 \linewidth} p{0.15 \linewidth}}
\toprule
Description & PDB ID & Experiments & Experimental references & Computational references \\
\midrule
c-Myb transactivation domain & 1SB0 & \ce{^1H} chemical shifts &  &  \\
& & NOESY & & \\
Short peptides & & HDX exchange rates &  &  \\
\bottomrule
\end{tabular}
\label{tab:nmr}
\end{table*}

\section{Hydrogen-deuterium exchange (HDX) experiments}

\begin{itemize}

\item Advantages of HDX

	\begin{itemize}
	\item HDX informs on folding of small proteins with simple tertiary structures
	\item HDX discriminates between proteins with intermediate and high folding stability that have similar bulk properties or spin relaxation rates
	\end{itemize}

\item Observables

	\begin{itemize}
	\item Chemical shifts or HSQC measured by NMR
	\item Mass spectrometry
	\item Protection factor (exchange frequency relative to unfolded state) has an ambiguous relationship to computable quantities, e.g. free energies
	\end{itemize}

\item Systems

	\begin{itemize}
	\item Gabe Rocklin and Tobin Sosnick HDX dataset
	\item Vincent Shaw G proteins
	\item Vincent Voelz ubiquitin, BPTI, and myoglobin
	\end{itemize}

\end{itemize}

\section{List of potential figures}

\begin{itemize}

\item Visualization of protein crystal supercell

\item Visualization of differences in electron density with $F_O - F_C$ map

\item Solution protein structure with NMR observables labeled

	\begin{itemize}
	\item Folded tertiary structure labeled with "RDC" and "Spin relaxation"
	\item Long range contact labeled with "NOE"
	\item Inset of $\alpha$ helix labeled with "HDX" and "Helical propensity"
	\item Inset of peptide backbone with "Chemical shift" and "$^3J$ coupling" labeled
	\end{itemize}

\item Histograms of observables in larger datasets (perhaps borrowed from original publications

	\begin{itemize}
	\item Distribution of spin relaxation rates in Ollila dataset
	\item Distribution of HDX exchange rates in Rocklin/Sosnick dataset
	\end{itemize}

\end{itemize}

\section{Conclusions}

\begin{itemize}

\item Summarize key points

\item Additional type of experiments

	\begin{itemize}
	\item Kirkwood-Buff integrals for co-solvents
	\item Paramagnetic relaxation enhancement interactions
	\item Binding free energies
	\item Salt bridge dissociation rates
	\item Folding observables

		\begin{itemize}
		\item Free energies
		\item Kinetic rates
		\item Melting temperatures
		\end{itemize}

	\item Small angle x-ray scattering observables

		\begin{itemize}
		\item Radii of gyration
		\item Kratky plots
		\item Pairwise distribution functions
		\end{itemize}

	\end{itemize}

\item Additional protein systems

	\begin{itemize}
	\item Membrane proteins (Benoit Roux)
	\item Proteins with ligands or cofactors
	\item Protein mimetics, e.g. peptoids or $\beta$-peptides (Lillian Chong)
	\end{itemize}

\end{itemize}

\section{Author Contributions}
%%%%%%%%%%%%%%%%
% This section mustt describe the actual contributions of
% author. Since this is an electronic-only journal, there is
% no length limit when you describe the authors' contributions,
% so we recommend describing what they actually did rather than
% simply categorizing them in a small number of
% predefined roles as might be done in other journals.
%
% See the policies ``Policies on Authorship'' section of https://livecoms.github.io
% for more information on deciding on authorship and author order.
%%%%%%%%%%%%%%%%

(Explain the contributions of the different authors here)

% We suggest you preserve this comment:
For a more detailed description of author contributions,
see the GitHub issue tracking and changelog at \githubrepository.

\section{Other Contributions}
%%%%%%%%%%%%%%%
% You should include all people who have filed issues that were
% accepted into the paper, or that upon discussion altered what was in the paper.
% Multiple significant contributions might mean that the contributor
% should be moved to authorship at the discretion of the a
%
% See the policies ``Policies on Authorship'' section of https://livecoms.github.io for
% more information on deciding on authorship and author order.
%%%%%%%%%%%%%%%

(Explain the contributions of any non-author contributors here)
% We suggest you preserve this comment:
For a more detailed description of contributions from the community and others, see the GitHub issue tracking and changelog at \githubrepository.

\section{Potentially Conflicting Interests}
%%%%%%%
%Declare any potentially competing interests, financial or otherwise
%%%%%%%

MKG has an equity interest in and is a cofounder and scientific advisor of VeraChem.

\section{Funding Information}
%%%%%%%
% Authors should acknowledge funding sources here. Reference specific grants.
%%%%%%%
We thank the Open Force Field Consortium and Initiative for financial and scientific support and the Molecular Sciences Software Institute (MolSSI) for its support of the Open Force Field Initiative.
We also thank the National Institutes of Health (NIH) for support of this research via NIH R01GM132386.
MKG acknowledges funding from National Institute of General Medical Sciences (GM061300).
These findings are solely of the authors and do not necessarily represent the views of the NIH.

\section*{Author Information}
\makeorcid

\bibliography{protein-benchmark-data}

%%%%%%%%%%%%%%%%%%%%%%%%%%%%%%%%%%%%%%%%%%%%%%%%%%%%%%%%%%%%
%%% APPENDICES
%%%%%%%%%%%%%%%%%%%%%%%%%%%%%%%%%%%%%%%%%%%%%%%%%%%%%%%%%%%%

%\appendix


\end{document}
